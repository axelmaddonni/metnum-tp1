\subsection{Apéndice A}

{\bf Enunciado}

Se debe implementar un programa en \verb+C+ o \verb-C++- que tome como entrada los par\'ametros del problema ($r_i$, $r_e$, $m+1$,
$n$, valor de la isoterma buscada, $T_i$, $T_e(\theta)$) que calcule la temperatura dentro de la pared del horno utilizando el
modelo propuesto en la secci\'on anterior y que encuentre la isoterma buscada en funci\'on del resultado obtenido del
sistema de ecuaciones. El m\'etodo para determinar la posici\'on de la isoterma queda a libre elecci\'on de cada grupo y
debe ser explicado en detalle en el informe.

El programa debe formular el sistema obtenido a partir de las ecuaciones (1) - (6) y considerar dos m\'etodos posibles
para su resoluci\'on: mediante el algoritmo cl\'asico de Eliminaci\'on Gaussiana y la Factorizaci\'on LU. Finalmente, el
programa escribir\'a en un archivo la soluci\'on obtenida con el formato especificado en la siguiente secci\'on.

Como ya se ha visto en la materia, no es posible aplicar los m\'etodos propuestos para la resoluci\'on a cualquier
sistema de ecuaciones. Sin embargo, la matriz del sistema considerado en el presente trabajo cumple con ser diagonal dominante (no
estricto) y que, ordenando las variables y ecuaciones convenientemente, es posible armar un sistema de ecuaciones cuya matriz
posee la propiedad de ser \emph{banda}. Luego, se pide demostrar (o al menos dar un esquema de la demostraci\'on)
el siguiente resultado e incluirlo en el informe:

\begin{proposition}
Sea $A \in \mathbb{R}^{n \times n}$ la matriz obtenida para el sistema definido por (1)-(6). Demostrar que es posible
aplicar Eliminaci\'on Gaussiana sin pivoteo.\footnote{Sugerencia: Notar que la matriz es diagonal dominante (no
estrictamente) y analizar qué sucede al aplicar un paso de Eliminaci\'on Gaussiana con los elementos de una fila.} 
\end{proposition}

La soluci\'on del sistema de ecuaciones permitir\'a saber la temperatura en los puntos de la discretizaci\'on. Sin embargo,
nuestro inter\'es es calcular la isoterma 500, para poder determinar si la estructura se encuentra en peligro. Luego, se pide lo siguiente:
\begin{itemize}
\item Dada la soluci\'on del sistema de ecuaciones, proponer una forma de estimar en cada \'angulo de la discretizaci\'on la posici\'on de la 
isoterma 500.
\item En funci\'on de la aproximaci\'on de la isoterma, proponer una forma (o medida) a utilizar para evaluar la peligrosidad de la estructura
en funci\'on de la distancia a la pared externa del horno.
\end{itemize}


En funci\'on de la experimentaci\'on, se busca realizar dos estudios complementarios: por un lado, analizar c\'omo se comporta el sistema y, por otro, 
cu\'ales son los requerimientos computacionales de los m\'etodos. Se pide como m\'inimo realizar los siguientes experimentos:
\begin{enumerate}
\item Comportamiento del sistema.
\begin{itemize}
\item Considerar al menos dos instancias de prueba, generando distintas discretizaciones para cada una de ellas y
comparando la ubicaci\'on de la isoterma buscada respecto de la pared externa del horno. Se sugiere presentar gr\'aficos
de temperatura o curvas de nivel para los mismos, ya sea utilizando las herramientas provistas por la c\'atedra o
implementando sus propias herramientas de graficaci\'on. 
\item Estudiar la proximidad de la isoterma buscada respecto de la pared exterior del horno en funci\'on de distintas 
granularidades de discretizaci\'on y las condiciones de borde. 
\end{itemize}
\item Evaluaci\'on de los m\'etodos.
\begin{itemize}
\item Analizar el tiempo de c\'omputo requerido para obtener la soluci\'on del sistema en funci\'on de la granularidad de 
la discretizaci\'on. Se sugiere presentar los resultados mediante gr\'aficos de tiempo de c\'omputo en funci\'on de alguna 
de las variables del problema.
\item Considerar un escenario similar al propuesto en el experimento 1. pero donde las condiciones de borde (i.e., $T_i$ y $T_e(\theta)$)
cambian en distintos instantes de tiempo. En este caso, buscamos obtener la secuencia de estados de la temperatura en
la pared del horno, y la respectiva ubicaci\'on de la isoterma especificada. Para ello, se considera una secuencia de $ninst$
vectores con las condiciones de borde, y las temperaturas en cada estado es la soluci\'on del correspondiente sistema de
ecuaciones. Se pide formular al menos un experimento de este tipo, aplicar los m\'etodos de resoluci\'on propuestos de
forma conveniente y compararlos en t\'erminos de tiempo total de c\'omputo requerido para distintos valores de $ninst$.
\end{itemize}
\end{enumerate}

De manera opcional, aquellos grupos que quieran ir un poco m\'as all\'a pueden considerar trabajar y desarrollar alguno(s) 
de los siguientes puntos extra:
\begin{enumerate}
\item Notar que el sistema resultante tiene estructura \emph{banda}. Proponer una estructura para aprovechar este hecho en t\'erminos de la
\emph{complejidad espacial} y como se adaptar\'ian los algoritmos de Eliminaci\'on Gaussiana y Factorizaci\'on LU para reducir la
cantidad de operaciones a realizar.
\item Implementar dicha estructura y las adaptaciones necesarias para el algoritmo de Eliminaci\'on Gaussiana.
\item Implementar dicha estructura y las adaptaciones necesarias para el algoritmo de Factorizaci\'on LU. 
\end{enumerate}

Finalmente, se deber\'a presentar un informe que incluya una descripci\'on detallada de los m\'etodos implementados y
las decisiones tomadas, el m\'etodo propuesto para el c\'alculo de la isoterma buscada y los experimentos realizados,
junto con el correspondiente an\'alisis y siguiendo las pautas definidas en el archivo \verb+pautas.pdf+.


\subsection{Apéndice B}

\textbf{Código fuente para la creación de la matriz}
\textbf{}

\lstset{language=C++, breaklines=true, basicstyle=\footnotesize}
\begin{lstlisting}[frame=single]
vector< vector<double> > crearMatriz (double n, double m, double ri, double deltaR, double deltaA){

	vector< vector<double> > matriz((m+1)*n, vector<double>((m+1)*n)); 

	//Completo primeras n filas y ultimas n filas con 1s en la diagonal
	for (int i = 0; i < n; i++){
		matriz[i][i] = 1;
	}
	for (int i = (n*(m+1) - n); i < n*(m+1); i++){
		matriz[i][i] = 1;
	}

	//Completo resto de las ecuaciones usando las formulas
	for (int j = 1; j < m; j++){
		double r = ri + j * deltaR;
		for (int k = 0; k < n; k++){
			//uso para calcular en la matriz la posicion j,k = n x j + k

			// t j,k
			matriz[n*j + k][n*j + k] = (-2*pow(r,2)*pow(deltaA, 2) + r*deltaR*pow(deltaA, 2) - 2*pow(deltaR, 2))/(pow(deltaR, 2)*pow(r, 2)*pow(deltaA, 2));
			
			// t j-1, k
			matriz[n*j + k][n*(j-1) + k] = (r - deltaR)/(pow(deltaR, 2)* r);

			// t j+1, k
			matriz[n*j + k][n*(j+1) + k] = 1 / pow(deltaR, 2);

			// t j, k-1
			matriz[n*j + k][n*j + (k - 1 > 0? k - 1 : n - 1)] = 1 / (pow(r, 2)*pow(deltaA, 2));

			//t j, k+1
			matriz[n*j + k][n*j + (k + 1 < n - 1? k + 1 : 0)] = 1 / (pow(r, 2)*pow(deltaA, 2));
		}
	}
	return matriz;
}
\end{lstlisting}

\textbf{}
\textbf{Código fuente para Eliminación Gaussiana con aprovechamiento de banda}
\textbf{}

\lstset{language=C++, breaklines=true, basicstyle=\footnotesize}
\begin{lstlisting}[frame=single]
void EliminacionParaBanda (vector<vector<double>>& mat, int bandaAltura, int bandaAncho, vector<double>& adjunta)
{
    int h = mat.size();
    int w = mat[0].size();
    for (int PivoteDiagonal = 0; PivoteDiagonal < w; PivoteDiagonal++)
    {
        //corre desde la fila de abajo de la diagonal hasta la bandaAltura-iesima de abajo de la diagonal
        //el resto son 0 por la propiedad banda
        for (int FilaActual = PivoteDiagonal + 1; FilaActual < PivoteDiagonal + bandaAltura && FilaActual < h; FilaActual++)
        {
            if ( mat [FilaActual][PivoteDiagonal] != 0)
            {
                double coeficiente = mat [FilaActual][PivoteDiagonal] / mat [PivoteDiagonal][PivoteDiagonal];
                adjunta [FilaActual] -= adjunta[PivoteDiagonal] * coeficiente;
                //corre desde la columna de la diagonal hasta la bandaAncho-iesima de la derecha de la diagonal
                //el resto son 0 por la propiedad banda
                for (int ColumaEnSuma = PivoteDiagonal; ColumaEnSuma < FilaActual + bandaAncho && ColumaEnSuma < w; ColumaEnSuma++)
                {
                    mat [FilaActual][ColumaEnSuma] -= mat [PivoteDiagonal][ColumaEnSuma] * coeficiente;
                }
            }
        }
    }
}
\end{lstlisting}