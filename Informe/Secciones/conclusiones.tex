\par En primer lugar, concluimos (como se esperaba), que la Eliminaci\'on Gaussiana optimizada para banda es el algoritmo m\'as eficiente para resolver sistemas con pocos t\'erminos independientes distintos. Gauss result\'o ser m\'as eficiente que LU, pero por un factor muy peque\~no.
\par LU result\'o ser a\'un m\'as eficiente que Gauss optimizada para sistemas con muchos t\'erminos independientes distintos, a pesar de que ambos tienen complejidad cuadr\'atica (en el tama\~no de la matriz) despu\'es de la primera soluci\'on (en el caso de LU, la descomposici\'on original es c\'ubica).
\par La conclusi\'on m\'as interesante que obtuvimos fue respecto al efecto de las optimizaciones de banda: incluso las optimizaciones que originalmente consideramos menores para banda en Gauss y LU tienen un efecto muy significativo en el tiempo de ejecuci\'on de los algoritmos.
\par De estos comentarios se desprende la idea de que contar con cada detalle del sistema a modelar puede permitirnos plantear de mejor manera el problema y la forma en la que lo vamos a resolver, optimizando los resultados que luego se van a obtener. Por ejemplo plantear el sistema de ecuaciones de cierta forma nos permite aprovechar las características de una matriz banda y poder mejorar el rendimiento de nuestro programa.
\par También pudimos acercarnos mas a la investigación, sus formas y métodos, así como la idea de recibir colaboración por parte de especialistas para plantear las formas de resolver el problema, las ecuaciones y datos extras que pueden ayudarnos para dar con la solución y optimizar el proceso.