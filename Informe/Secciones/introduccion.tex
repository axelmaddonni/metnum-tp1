\subsection{Explicaci\'on de los m\'etodos utilizados}
\subsubsection{Eliminaci\'on Gaussiana}
\par El m\'etodo de Eliminaci\'on Gaussiana se utiliza para triangular superiormente una matriz. Esto se logra iterando sobre filas, utilizando un pivote diagonal (que llamaremos $i$ en la siguiente explicaci\'on por claridad); en cada iteraci\'on del ciclo exterior se resta a cada una de las filas inferiores a la i-\'esima, la i-\'esima multiplicada por un coeficiente tal que el elemento de la i-\'esima columna quede en 0. Al iterar sobre todas las filas (excepto la \'ultima, ya que no tiene filas inferiores a ella), se triangula la matriz. Al restarle una fila a otra, tambi\'en se restan los t\'erminos independientes correspondientes a cada fila de la misma forma.
\par Una optimizaci\'on que incorporamos es no restar las filas si el elemento a dejar en 0 ya est\'a en 0. Como la matriz que representa el problema es Banda (esta proposici\'on se encuentra demostrada en la secci\'on \ref{subsec:DemBanda}), y por ende deber\'ia tener una significante cantidad de valores en 0, esta optimizaci\'on deber\'ia ser considerablemente efectiva.
\subsubsection{Descomposici\'on LU}
\par La Descomposici\'on LU es similar a la Eliminaci\'on Gaussiana, pero al restarle la fila i-\'esima multiplicada por el coeficiente ($i$ siendo el pivote diagonal detallado en la secci\'on anterior) a la j-\'esima ($j$ siendo el \'indice de la fila sobre la cual se itera en el ciclo interno), se guarda tal coeficiente como el $j,i$-\'esimo valor de una nueva matriz. Los valores de la diagonal de la nueva matriz se inicializan en 1, y el resto en 0, y por la forma en que se itera para triangular superiormente una matriz, \'esta queda triangular inferior. 
\par La diferencia principal de la Descomposici\'on LU respecto de la Eliminaci\'on Gaussiana es que no involucra al t\'ermino independiente, y una vez descompuesta una matriz, se puede resolver el sistema para cualquier t\'ermino independiente. Ya que la complejidad de resolver un sistema triangular (o dos, en el caso de LU) es $O(n^2)$ ($n$ siendo el tama\~no de la matriz), y el de la Descomposici\'on LU o Gauss es $O(n^3)$, si se debe resolver el mismo sistema con distintos t\'erminos independientes, LU deber\'ia ser el m\'etodo superior, ya que s\'olo se debe pagar una vez el coste c\'ubico.
\par Utilizamos la misma optimizaci\'on para matriz banda con LU que mencionamos en la secci\'on anterior.
\subsubsection{Resolver un sistema triangular}
\par Para resolver un sistema triangular se itera por filas (comenzando a partir de la fila con un \'unico valor no-nulo), y se despeja el valor de la diagonal, rest\'andole al t\'ermino independiente todos los otros coeficientes por el valor de sus variables (que, si se itera correctamente, ya deber\'ian haber sido calculadas). Resolver un sistema triangular inferior es similar a resolver un sistema triangular superior, pero se comienza a iterar en distintos puntos y se deben despejar los valores de distintos lados de la diagonal (ya que los sistemas son triangulares no hace falta despejar todos los valores de la fila, ya que se sabe de antemano que los de un lado son todos 0).
\subsubsection{Resolver un sistema LU}
\par Resolver un sistema que fue descompuesto mediante LU consiste en encontrar la soluci\'on de la matriz triangular inferior con el t\'ermino independiente original, y luego la matriz triangular superior utilizando como t\'ermino independiente a la soluci\'on reci\'en encontrada para la matriz inferior.
